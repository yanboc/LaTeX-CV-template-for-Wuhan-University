\documentclass[11pt]{article}
\usepackage{xltxtra}
\usepackage{bookmark}
\usepackage{hyperref}
\hypersetup{hidelinks}
\usepackage{url}
\urlstyle{tt}
\usepackage{multicol}
\usepackage{xcolor}
\usepackage{calc}
\usepackage{graphicx}
\usepackage{tikz}
\usetikzlibrary{calc}
\usepackage{fontspec}
\usepackage{xeCJK}
\usepackage{relsize}
\usepackage{xspace}
\usepackage{fontawesome5}
\usepackage{titlesec}
\usepackage{enumitem}
\usepackage{siunitx}
\usepackage{amssymb}
\usepackage{tabularx}
\usepackage{multicol}
\usepackage{fontspec}
\usepackage{fancybox}
\usepackage{float}

% 一些小设置,参考自https://github.com/LeyuDame/BNUCV/tree/main
% 取消中文字符与数字之间的间隔
\CJKsetecglue{}
\protected\def\Cpp{{C\nolinebreak[4]\hspace{-.05em}\raisebox{.28ex}{\relsize{-1}++}}\xspace}	% 这是个更好看的C++写法,你直接写C++的话,+号会很大,可以使用\Cpp来代替
\setlength{\parindent}{0pt}							% 取消全局段落缩进
\pagenumbering{gobble}								% 取消页码显示
%\setlist{noitemsep}									% 禁用列表中项目之间的额外垂直间距,但保留列表周围的间距
%\setlist{nosep}										% 禁用列表中项目之间的额外垂直间距及列表周围的间距
\setlist[itemize]{topsep=0em, leftmargin=*}		% 增加了itemize顶部间距
\setlist[enumerate]{topsep=0em, leftmargin=*}	% 增加了enumerate顶部间距

\titleformat{\section}					    % 将原标题前面的数字取消了
  {\LARGE\bfseries\raggedright} 		      % 字体改为LARGE,bold,左对齐
  {}{0em}                      			  % 可用于添加全局标题前缀
  {}                           			  % 可用于添加代码
  [{\color{WHU_Blue}\titlerule}]            % 标题下方加一条线
\titlespacing*{\section}{0cm}{*1.2}{*1.2}	% 标题左边留白,上方1.2倍,下方1.2倍

\titleformat{\subsection}				    % 将原二级标题前面的数字取消了
  {\large\bfseries\raggedright} 		      % 字体改为large,bold,左对齐
  {}{0em}                      			  % 可用于添加全局二级标题前缀
  {}                           			  % 可用于添加代码
  []
\titlespacing*{\subsection}{0cm}{*1.2}{*1.2}% 二级标题左边留白,上方1.2倍,下方1.2倍
% 页面大小与页边距,按需求调整
\usepackage[
	a4paper,
	left=1.2cm,
	right=1.2cm,
	top=1.5cm,
	bottom=1cm,
	nohead
]{geometry}
% 这里把表格的行间距调成1.2倍了
\renewcommand{\arraystretch}{1.2}
% 这里把正文的行间距调成1.2倍了
\linespread{1.2}
% 中文字符间距
\renewcommand{\CJKglue}{\hskip 0.05em}

% 自定义宋体(狮尾四季春),详见readme
% 英文字体
\setmainfont[
    Path=fonts/,
    Extension=.ttf,
    BoldFont=* Bold,
]{SweiSpring}
% 中文字体
\setCJKmainfont[
    Path=fonts/,
    Extension=.ttf,
    BoldFont=* Bold,
]{SweiSpring}

% 自定义主题色
% 武大蓝(来自官方配色方案,参考https://www.whu.edu.cn/xxgk/wdbs.htm)
\definecolor{WHU_Blue}{RGB}{0, 37, 84}



% 填写个人信息
% 学院
\newcommand{\school}{拓新研究院 | School of Tuoxin} 

% 联系方式
\newcommand{\contact}{
    % 根据个人喜好选择字号
    % \small              % 小
    % \footnotesize       % 更小
    \scriptsize         % 再小一号
    \textcolor{white}{
        % 邮箱
        \faEnvelope \quad \href{mailto:youremail@whu.edu.com}{youremail@whu.edu.com}
        \hspace{4em}
        % 手机号
        \faPhone \quad  130-6666-0000
        % 别的联系方式,如微信、GitHub等
        % \hspace{4em}
        % \faGithub \quad \href{https://github.com/xxxx}{https://github.com/xxxx}
    }
}



%%%%%%%%%%%%%%%%%%%%
% 简历正文
%%%%%%%%%%%%%%%%%%%%
\begin{document}
% 如果有多页简历,请把页眉页脚和背景复制粘贴到第二页的内容之前
    % 页眉:校标组合+学院名
    \begin{tikzpicture}[remember picture, overlay]
        \node[anchor=north, inner sep=0pt](header) at (current page.north){
            \includegraphics[width=\paperwidth]{images/header.png}
        };
        \node[anchor=west](school_logo) at (header.west){
            \hspace{0.5cm}
            \includegraphics[width=0.15\textwidth]{images/whu_logo_1.png}
        };
        \node[anchor=east](school_name) at(header.east){
            \textcolor{white}{\textbf{\school}}
            \hspace{0.5cm}
        };
    \end{tikzpicture}
    \vspace{-3.5em}

    % 页脚,联系方式
    \begin{tikzpicture}[remember picture, overlay]
        \node[anchor=south, inner sep=0pt](footer) at (current page.south){
            \includegraphics[width=\paperwidth]{images/footer.png}
        };
        % 联系方式
        \node[anchor=center] at(footer.center){\contact};
    \end{tikzpicture}

    % 背景
    \begin{tikzpicture}[remember picture, overlay]
        \node[opacity=0.05] at(current page.center){
            \includegraphics[width=0.7\paperwidth, keepaspectratio]{images/whu_logo_big.png}
        };
    \end{tikzpicture}

    % 个人信息
    \begin{figure}[h]
        % 左半边,信息,比例占行宽82%,可以自己调
        \begin{minipage}{0.82\textwidth}
            \section{\makebox[\widthof{\faAddressCard}][c]{\color{WHU_Blue}{\faAddressCard}}\quad 个人信息}
            \begin{tabularx}{\linewidth}{p{\widthof{出生日期:}}Xp{\widthof{政治面貌:}}X}
                姓\ \ \ \ \ \ \ \ 名: & 你的名字 & 
                性\ \ \ \ \ \ \ \ 别: & 你的性别  \\
                出生年月: & 你的出生年月 & 
                政治面貌: & 你的政治面貌 \\
                %% 想多加几行的话,就按上面的格式自行补充
                %% 想加粗的话\textbf{}
                %% 想多加几列的话,把\begin{tabularx}{\textwidth}{这里}的内容改一下,可以自己搜一下tabularx怎么用,也可以问gpt/文心一言/讯飞。
            \end{tabularx}
        \end{minipage}
    \hspace{2em}
    % 右半边,照片,比例占行宽12%,可以自己调
    % images/avatar.png 替换成你证件照的路径。
    \begin{minipage}{0.12\textwidth}
        \setlength{\fboxsep}{0pt}
        \doublebox{\includegraphics[width=\linewidth]{images/avatar.png}}
    \end{minipage}
    \end{figure}
    \vspace{-1em}

    % 教育背景
    % \faGraduationCap这类\fa开头的都是font awesome里的logo,想换成其他logo的话,可以看一下附带的fontawsome.pdf,自行替换。
    \section{\makebox[\widthof{\faGraduationCap}][c]{\color{WHU_Blue}{\faGraduationCap}}\quad 教育背景}

    % 教育背景(本科生)
    % \vspace{-1em}
    % \begin{table}[h!]
    %     \begin{tabularx}{\textwidth}{XXp{\widthof{2021年 -- 预计2025年7月毕业}}}
    %         武汉大学 & 电子信息工程 & 2021年 -- 预计2025年7月毕业\\
    %         \textbf{GPA: 4.0/4.0} & \textbf{GPA排名: 1/100} & \textbf{综测排名: 1/100} \\
    %     \end{tabularx}
    % \end{table}

    % 教育背景(研究生)
    {\large \textbf{武汉大学}},本科 \hfill {湖北,武汉} \\
    \href{学院官网.whu.edu.cn}{\underline{自强学堂}},专业:你的专业 \hfill {2000年9月-2010年6月} \\
    {主修课程}:课程1、课程2、课程3、课程4\ 等。

    \vspace{0.5em}
    {\large \textbf{武汉大学}},硕士 \hfill {湖北,武汉} \\
    {{求是学院}},专业:基础数学 \hfill {2010年9月-2020年6月} \\
    \textbf{主修课程}:课程1、课程2、课程3、课程4\ 等。

    \vspace{0.5em}
    {\large \textbf{武汉大学}},博士 \hfill {湖北,武汉} \\
    {{拓新研究院}},导师:\href{导师的个人主页.site}{导师名字}\ 导师职称 \hfill {2020年9月-至今} \\
    \textbf{研究方向}:方向1、方向2、方向3、方向4\ 等。

    % 教育背景
    \section{\makebox[\widthof{\faGraduationCap}][c]{\color{WHU_Blue}{\faGraduationCap}}\quad 科研成果}


    % 科研著作(研究生)
    This is One of Your Paper Published in Conference A. \\
    \textbf{Mingzi Nide}, Daoshi Nide. \hfill 
    发表于 \textbf{Conference A}(CCF-A类会议) 

    \vspace{0.5em}
    This is Another Paper. \\
    \textbf{Mingzi Nide}, Shidi Nide, Daoshi Nide. \hfill 
    发表于 \textbf{Conference B} (CCF-A类会议)

    \vspace{0.5em}
    This is A Journal Paper.\\
    \textbf{Mingzi Nide}, Shixiong Nide, Daoshi Nide. \hfill 
    发表于 \textbf{Journal C} (SCI-1区)

    % 项目经历\科研经历\项目与教学(标题请根据需要修改)
    \section{\makebox[\widthof{\faChalkboardTeacher}][c]{\color{WHU_Blue}{\faChalkboardTeacher}}\quad 项目与教学}
    \vspace{0.5em}
    {\large{\textbf{项目名称}}} \hfill {横向/纵向项目-已完结/进行中}\\
    \textbf{你在项目中扮演的角色} \hfill 2020年9月至2021年9月\\
    项目简介。

    \vspace{1em}
    {\large{\textbf{某某主题讨论班}}},主讲/参与 \hfill {2020年夏季}\\
    主要内容:内容1,内容2,内容3\ 等。
    
    \vspace{1em}
    {\large{\textbf{课程名称}}},助教 \hfill {2021年夏季}\\
    主要内容:内容1,内容2,内容3\ 等。

    % 技能特长(标题根据个人需求修改)
    \section{\makebox[\widthof{\faWrench}][c]{\color{WHU_Blue}{\faWrench}}\quad 技能}
    \vspace{0.5em}
    \begin{itemize}
        \item 英语:六级800分、托福200分;
        \item 编程:Python, R, MATLAB, C.
    \end{itemize}
    
    % \newpage
    % 如有需要,再加一页。可以写荣誉、竞赛等。 

    % 页眉页脚不要删。
    % % 页眉:校标组合+学院名
    % \begin{tikzpicture}[remember picture, overlay]
    %     \node[anchor=north, inner sep=0pt](header) at (current page.north){
    %         \includegraphics[width=\paperwidth]{images/header.png}
    %     };
    %     \node[anchor=west](school_logo) at (header.west){
    %         \hspace{0.5cm}
    %         \includegraphics[width=0.15\textwidth]{images/whu_logo_1.png}
    %     };
    %     \node[anchor=east](school_name) at(header.east){
    %         \textcolor{white}{\textbf{\school}}
    %         \hspace{0.5cm}
    %     };
    % \end{tikzpicture}
    % \vspace{-4em}
    
    % % 页脚,联系方式
    % \begin{tikzpicture}[remember picture, overlay]
    %     \node[anchor = south, inner sep=0pt] at (current page.south){
    %         \includegraphics[width=\paperwidth]{images/footer.png}
    %     };
    %     % 联系方式
    %     \node[anchor=center] at(footer.center){\contact};
    % \end{tikzpicture}
    
    % % 背景
    % \begin{tikzpicture}[remember picture, overlay]
    %     \node[opacity=0.1] at(current page.center){
    %         \includegraphics[width=0.7\paperwidth, keepaspectratio]{images/whu_logo_big.png}
    %     };
    % \end{tikzpicture}


    % % 竞赛经历
    % \section{\makebox[\widthof{\faTrophy}][c]{\color{WHU_Blue}{\faTrophy}}\quad 竞赛经历}
    % \vspace{-1em}
    % \begin{table}[h!]
    %     \begin{tabularx}{\textwidth}{Xp{\widthof{第零负责人}}p{\widthof{国家级-第100名}}p{\widthof{2030年13月}}}
    %         \textbf{比赛1} & 第一负责人 & 国家级-第10名 & 2023年4月 \\
    %         \textbf{比赛2} & 个人参赛 & 国家级-一等奖 & 2023年8月\\
    %         \textbf{比赛3} & 个人参赛 & 省级-一等奖 & 2022年12月\\
    %         % 同理,可以自己加
    %     \end{tabularx}
    % \end{table}

    % % 技能特长
    % \section{\makebox[\widthof{\faWrench}][c]{\color{WHU_Blue}{\faWrench}}\quad 技能特长}
    % \vspace{0.5em}
    % \begin{itemize}
    %     \item 熟练使用\Cpp 、Python、Matlab编程语言。
    %     \item 熟悉Windows与Linux端开发。
    %     \item 熟练使用Tensorflow,Pytorch等深度学习框架。
    %     \item 熟练掌握\Cpp 与Python环境下OpenCV与Qt应用的开发,且熟练使用Qt Creator软件。
    %     \item 熟练使用Altium Designer与LCEDA进行封装绘制与板子设计。
    %     \item 熟练使用Keil,Arduino IDE等集成开发软件。
    %     \item 了解模式识别,强化学习,遗传算法,知识蒸馏等相关概念。
    % \end{itemize}

    % % 所获荣誉
    % \section{\makebox[\widthof{\faStar}][c]{\color{WHU_Blue}{\faStar}}\quad 所获荣誉}
    % \vspace{-1em}
    % \begin{multicols}{2}
    %     \begin{itemize}
    %         \item 某年学业先进个人
    %         \item 某年某奖学金某等奖
    %         \item 某大使
    %         \item 某年某奖学金某等奖
    %         \item 某年优秀团员称号
    %         \item 某年某称号
    %     \end{itemize}
    % \end{multicols}

    % % 其他
    % \section{\makebox[\widthof{\faInfo}][c]{\color{WHU_Blue}{\faInfo}}\quad 其他}
    % \begin{itemize}
    %     \item 英语水平-CET6级xxx分
    %     \item 计算机几级证书
    %     \item xx几级证书
    %     \item 技术博客: 某网址
    %     \item 教师资格证:xxx
    %     \item 普通话证书:几级几等
    %     \item 文字排版:\LaTeX
    % \end{itemize}

\end{document}
